% Options for packages loaded elsewhere
\PassOptionsToPackage{unicode}{hyperref}
\PassOptionsToPackage{hyphens}{url}
%
\documentclass[
]{article}
\usepackage{amsmath,amssymb}
\usepackage{lmodern}
\usepackage{iftex}
\ifPDFTeX
  \usepackage[T1]{fontenc}
  \usepackage[utf8]{inputenc}
  \usepackage{textcomp} % provide euro and other symbols
\else % if luatex or xetex
  \usepackage{unicode-math}
  \defaultfontfeatures{Scale=MatchLowercase}
  \defaultfontfeatures[\rmfamily]{Ligatures=TeX,Scale=1}
\fi
% Use upquote if available, for straight quotes in verbatim environments
\IfFileExists{upquote.sty}{\usepackage{upquote}}{}
\IfFileExists{microtype.sty}{% use microtype if available
  \usepackage[]{microtype}
  \UseMicrotypeSet[protrusion]{basicmath} % disable protrusion for tt fonts
}{}
\makeatletter
\@ifundefined{KOMAClassName}{% if non-KOMA class
  \IfFileExists{parskip.sty}{%
    \usepackage{parskip}
  }{% else
    \setlength{\parindent}{0pt}
    \setlength{\parskip}{6pt plus 2pt minus 1pt}}
}{% if KOMA class
  \KOMAoptions{parskip=half}}
\makeatother
\usepackage{xcolor}
\usepackage[margin=1in]{geometry}
\usepackage{graphicx}
\makeatletter
\def\maxwidth{\ifdim\Gin@nat@width>\linewidth\linewidth\else\Gin@nat@width\fi}
\def\maxheight{\ifdim\Gin@nat@height>\textheight\textheight\else\Gin@nat@height\fi}
\makeatother
% Scale images if necessary, so that they will not overflow the page
% margins by default, and it is still possible to overwrite the defaults
% using explicit options in \includegraphics[width, height, ...]{}
\setkeys{Gin}{width=\maxwidth,height=\maxheight,keepaspectratio}
% Set default figure placement to htbp
\makeatletter
\def\fps@figure{htbp}
\makeatother
\setlength{\emergencystretch}{3em} % prevent overfull lines
\providecommand{\tightlist}{%
  \setlength{\itemsep}{0pt}\setlength{\parskip}{0pt}}
\setcounter{secnumdepth}{-\maxdimen} % remove section numbering
\ifLuaTeX
  \usepackage{selnolig}  % disable illegal ligatures
\fi
\IfFileExists{bookmark.sty}{\usepackage{bookmark}}{\usepackage{hyperref}}
\IfFileExists{xurl.sty}{\usepackage{xurl}}{} % add URL line breaks if available
\urlstyle{same} % disable monospaced font for URLs
\hypersetup{
  pdftitle={Request \#: 521 - Other - Dissertation},
  pdfauthor={McKenna Voorhees {[}A01769980{]} - Doc Student (w/Dr.~Heidi Wengreen)},
  hidelinks,
  pdfcreator={LaTeX via pandoc}}

\title{Request \#: 521 - Other - Dissertation}
\usepackage{etoolbox}
\makeatletter
\providecommand{\subtitle}[1]{% add subtitle to \maketitle
  \apptocmd{\@title}{\par {\large #1 \par}}{}{}
}
\makeatother
\subtitle{An analysis of food access, food insecurity, and food program
utility in the context of COVID-19 among SNAP-eligible Utahns}
\author{McKenna Voorhees {[}A01769980{]} - Doc Student (w/Dr.~Heidi
Wengreen)}
\date{February 04, 2021}

\begin{document}
\maketitle

\hypertarget{background}{%
\subsubsection{Background}\label{background}}

Please see Service Request section for more details.

Study aims are attached at the bottom for the first two dissertation
project ideas.

\hypertarget{sample}{%
\subsubsection{Sample}\label{sample}}

Data has been collected via Qualtrics for all three surveys we hope to
include in the dissertation. Topic 1: interprofessional perceptions of
dietitians and perceived quality of interprofessional education in
dietetics programs.--Have \{\char`\textasciitilde\}800 responses

Topic 2: Food insecurity/food access outcomes as a result of COVID-19 in
SNAP-eligible Utahns. Have \{\char`\textasciitilde\}520 responses

Topic 3: Assessing food insecurity among individuals with disabilities
from an interdisciplinary standpoint. Responses are unknown.

\hypertarget{hypothesis}{%
\subsubsection{Hypothesis}\label{hypothesis}}

Topic 1 Research Objectives: The first objective of this study is to
examine associations between the characteristics of dietetic
students/practicing RDs and their perceptions of interprofessional
healthcare teams. Student characteristics that will be examined will
include program location and type, student career interests, and year of
study. RD characteristics that will be explored will include area of
specialty, time with credential, location of practice, gender, and age.
The second objective of this research is to analyze differences in the
perceptions of interprofessional healthcare teams between students
working towards attaining the credentials to become an RD and
professional RDs currently in practice. The third objective of this
study is to comprehensively explore ways in which dietetics programs
nation-wide meet accreditation curriculum requirements involving
interprofessional education . Other factors to be investigated include
evaluation methods of interprofessional student learning, when the topic
is primarily addressed, perceived effectiveness of current efforts, and
the amount of time spent on interprofessional education.

Topic 2 Research objectives: Objective \{\#\}1: The first objective of
this research was to evaluate SNAP-eligible participants' perceptions of
various food programs as a result of the onset of the COVID-19 outbreak.
Specific associations and differences examined include barriers existing
within each program and participation in food programs as a whole;
challenges and worry relating to general food access; and proposed
personal and system approaches determined to be helpful in addressing
noted challenges.

Sub-aim \{\#\}1: Determine whether fruit and vegetable intake was
significantly different among Utahns participating in various federal
and state-funded nutrition programs (e.g.~WIC, School Food program,
SNAP) with regard to the timing of program participation and COVID-19.

Objective \{\#\}2: The second objective of this research was to explore
the relationship of food insecurity status during the COVID-19 pandemic
and attitudes encompassing 'difficulties, compensatory strategies, and
proposed interventions among SNAP-eligible Utahns.

Sub-aim \{\#\}1: Assess whether associations exist among SNAP-eligible
participants' food insecurity status and fruit and vegetable intake.

Sub-aim \{\#\}2: Determine if food insecurity status is associated with
increased food access concerns in individuals those with special dietary
needs.

Objective \{\#\}3: The third objective of this study was to determine
SNAP-eligible Utahns' utilization of SNAP-Ed resources, including
involvement in SNAP-Ed classes and the following of online platforms. An
analysis of potential correlations between resource utilization and food
insecurity status and fruit and vegetable consumption, among others, was
also conducted.

Topic 3 Research objectives--not yet specifically defined.

\hypertarget{progress}{%
\subsubsection{Progress}\label{progress}}

I have completed statistical analyses on topics 1 \{\&\} 2, but no data
cleaning, exploratory summaries, or anything else for topic 3.

\hypertarget{request}{%
\subsubsection{Request}\label{request}}

See above. With COVID, a new project opportunity arose and grew. We
would like to find a way to incorporate it into my dissertation, but the
first topic of my dissertation, assessing interprofessional perceptions
in clinical Registered Dietitians and surveying quality of
interprofessional education in dietetics programs nation-wide, is quite
different from the second topic. The second topic is on the impact of
COVID-19 to Supplemental Nutrition Assistance Program (SNAP)-eligible
individuals in Utah as it relates to food access and food insecurity. We
have a third idea in mind that broadly connects the two topics; it was
funded by a USDA grant and the overarching objective was to look at food
insecurity in individuals with disabilities from an interdisciplinary
lens. The grant abstract is included in the attachments. We thought it
might make most sense to connect the three topics through a
methodological approach, but aren't quite sure how to do that and if we
do figure out something, what I need to change about the current data
analysis I have done. I have not done any analysis for the third
project, but the first project (interprofessional teams), I did a linear
regression approach to examine predictors of interprofessional
perceptions. The Attitudes Toward Interdisciplinary Team Scale was used
for this as the dependent variable. For the second project (on food
insecurity in SNAP-eligible Utahns as a result of COVID-19), there
wasn't a formal tool used as a dependent variable, so I used a
non-parametric approach (Kruskal Wallis H) for some of the Likert-style
questions. to look at associations and differences. Also used Chi-Square
Tests of Independence to look at categorical associations.

We conducted an EFA for the first survey/study in our sample; perhaps
validating or creating a survey for the second and/or third might be an
option?

I am not sure if this information is helpful, but all three utilized a
survey method.

Included all surveys aside from the Topic 3

\hypertarget{timeline}{%
\subsubsection{Timeline}\label{timeline}}

Ideally, we want to submit the first manuscript in March for
publication, and the third manuscript in late May or early June.
However, for purposes of the dissertation, these deadlines can be
modified where needed depending on what we decide is a good method for
integrating all three dissertation topics together.

\end{document}
